In conclusione, abbiamo mostrato come sia possibile utilizzare il meccanismo NEAT
per implementare soluzioni al problema presentato. Abbiamo descritto due
approcci diversi per fornire informazioni alla rete neurale. Tramite le reti
neurali è infatti possibile utilizzare informazioni sulla posizione del Robby
da integrare con le viste locali, oppure una vista globale della mappa. Tali
soluzioni risultano facilmente implementabili su reti neurali, ma introducono
problemi di scalabilità non banali per l'algoritmo genetico puro. Rispetto ai
risultati ottenuti, la valutazione non può che essere positiva per quanto
riguarda le viste locali, in quanto i valori di fitness ottenuti risultano
commisurati al numero di turni concessi ai Robby. A differenza della vista
locale la known map ha condotto a risultati deludenti essendo particolarmente
soggetta a massimi locali e \emph{plateau}. Nonostante i buoni risultati vi è
spazio per ulteriori miglioramenti, in particolare per integrare informazioni
globali che migliorino le performance.

Un approccio interessante rispetto al suddetto problema riguarda la possibilità
di collaborazione fra più di un Robby sulla stessa mappa. Questa modifica
consente di evidenziare eventuali interazioni complesse fra i Robby. Da questo
punto di vista è fondamentale individuare un paradigma di comunicazione
efficiente, in grado cioè di massimizzare lo scambio di informazioni nel modo
più sintetico possibile.\\

In questo contesto, l'approccio genetico puro risulta problematico:
aumentando il numero di Robby presenti sulla mappa, il numero di configurazioni
considerate dall'algoritmo genetico cresce in modo esponenziale, rendendo
computazionalmente dispendiosa la ricerca di corrispondenze efficaci fra
configurazioni in input e possibili azioni. Si è deciso pertanto di tentare
un approccio basato su reti neurali. Tale metodo consente infatti di non
mantenere la corrispondenza diretta tra dati in input e azione da intraprendere,
in quanto tali corrispondenze sono codificate direttamente all'interno della
rete. Fra i vari approcci presenti in letteratura, si è scelto di procedere
mediante una soluzione che evolva la rete in modo puramente genetico.
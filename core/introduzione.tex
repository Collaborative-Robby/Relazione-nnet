Il problema dell'esplorazione di mappe inizialmente sconosciute, finalizzata alla
localizzazione di determinati oggetti, è ben nota in letteratura. Tale problema
rientra nella branca dei problemi risolvibili tramite meccanismi di intelligenza
artificiale. Tale problema, come molti altri della stessa categoria, può essere
risolto tramite meccanismi descrittivi del problema, i quali codificano in un
algoritmo l'esatto comportamento da adottare, oppure meccanismi di apprendimento
i quali selezionano una tattica vincente tramite le performance attese. In
quest'ultima categoria rientrano, in particolare, gli approcci evolutivi genetici
e basati su reti neurali. I primi, ispirati alla teoria della selezione naturale
di Charles Darwin, evolvono il comportamento atteso selezionando i candidati in
base alla loro capacità di risolvere il problema in esame. Un aspetto cruciale
riguardante tale approccio è la rappresentazione del genoma, rappresentante le
caratteristiche di un singolo individuo, che viene usato per evolvere la specie
tramite riproduzione sessuata e mutazione in modo analogo a quanto avviene in
natura.\\

Gli approcci basati su reti neurali sono anch'essi ispirati alla biologia:
essi emulano difatti il comportamento del sistema nervoso centrale animale, nel
quale un segnale elettrico si propaga in una rete di neuroni
consentendo computazioni complesse. Al fine di tarare la rete, è possibile
utilizzare approcci supervisionati, che modificano la rete a posteriori a
partire dagli output desiderati, ed approcci non supervisionati, i quali
evolvono la rete in base alle performance della stessa nella risoluzione del
problema.\\

Sono comunque praticabili approcci ibridi, i quali sono in grado di evolvere
reti neurali tramite algoritmi genetici. Un aspetto cruciale negli approcci
basati su reti neurali con apprendimento non supervisionato o
apprendimento genetico puro è la funzione di \emph{fitness}, la quale deve non
solo codificare la bontà di un comportamento, ma anche guidare l'evoluzione
dell'intero sistema.

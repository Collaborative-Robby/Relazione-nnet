\begin{itemize}
 \item Modello
 \item Robby genetico
\end{itemize}
Il problema affrontato riguarda l'utilizzo di un robot, chiamato Robby, il cui 
scopo è quello di pulire una stanza dalla forma quadrata raccogliendo delle lattine.
La mappa entro la quale il Robby si muove consiste di una griglia 
bidimensionale di celle, che ammette solo posizioni intere.
Le lattine vengono quindi posizionate su celle inizialmente ignote al Robby. Si 
suppone che il Robby sia equipaggiato con sensori in grado di distinguere i muri 
che delimitano l'area esplorabile e di individuare l'eventuale presenza di lattine. La distanza 
massima che il Robby può osservare è un parametro variabile, ma è comunque 
limitata. Altrettanto limitato è il numero di passi discreti che il Robby può 
effettuare; l'azione di raccolta delle lattine viene considerata nel conteggio 
totale dei passi effettuati. La disposizione degli oggetti sulla mappa è 
statica, in altre parole non è possibile che compaiano
%(FIXME)
ulteriori lattine durante l'esecuzione.
\\
Rispetto ai possibili spostamenti, il Robby può muoversi solamente
in quattro direzioni, non vengono perciò considerate direzioni espresse in 
forma di angolo. Ciascun passo consente perciò al Robby di spostarsi fra due celle 
adiacenti.
\\
Una modalità possibile per affrontare il suddetto problema consiste 
nell'utilizzo di un approccio puramente genetico; si procede creando una 
mappatura fra le possibili configurazioni della vista e le mosse che il Robby 
può effettuare. Tale mappatura viene evoluta tramite crossover e mutazioni 
casuali, al fine di massimizzarne l'efficienza del Robby.

\documentclass[a4paper,10pt,abstracton]{scrartcl}
\usepackage[utf8]{inputenc}
\usepackage{cite}
\usepackage{hyperref}
\usepackage[italian]{babel}
\usepackage{float}
\usepackage{graphicx}
\usepackage{cleveref}
\usepackage{tikz}

\usetikzlibrary{arrows,shapes,positioning}

\begin{document}

\title{Robby the robot: un approccio collaborativo basato su reti neurali}
\subtitle{Progetto di Fisica dei Sistemi Complessi}

\author{Davide Berardi, Michele Corazza}

\maketitle

\begin{abstract}
parappapa
\end{abstract}

\section{Introduzione}
blablabla

\section{Scenario}

Il problema affrontato riguarda l'utilizzo di un robot, chiamato Robby, il cui 
scopo è quello di pulire una stanza dalla forma quadrata raccogliendo delle lattine.
La mappa entro la quale il Robby si muove consiste di una griglia 
bidimensionale di celle, che ammette solo posizioni intere.
Le lattine vengono quindi posizionate su celle inizialmente ignote al Robby. Si 
suppone che il Robby sia equipaggiato con sensori in grado di distinguere i muri 
che delimitano l'area esplorabile e di individuare l'eventuale presenza di lattine. La distanza 
massima che il Robby può osservare è un parametro variabile, ma è comunque 
limitata. Altrettanto limitato è il numero di passi discreti che il Robby può 
effettuare; l'azione di raccolta delle lattine viene considerata nel conteggio 
totale dei passi effettuati. La disposizione degli oggetti sulla mappa è 
statica, in altre parole non è possibile che compaiano
ulteriori lattine durante l'esecuzione.
\\
Rispetto ai possibili spostamenti, il Robby può muoversi solamente
in quattro direzioni, non vengono perciò considerate direzioni espresse in 
forma di angolo. Ciascun passo consente perciò al Robby di spostarsi fra due celle 
adiacenti.
\\
Una modalità possibile per affrontare il suddetto problema consiste 
nell'utilizzo di un approccio puramente genetico; si procede creando una 
mappatura fra le possibili configurazioni della vista e le mosse che il Robby 
può effettuare. Tale mappatura viene evoluta tramite crossover e mutazioni 
casuali, al fine di massimizzare l'efficienza del Robby.



\section {Obiettivi}
Un approccio interessante rispetto al suddetto problema riguarda la possibilità
di collaborazione fra più di un Robby sulla stessa mappa. Questa modifica
consente di evidenziare eventuali interazioni complesse fra i Robby. Da questo
punto di vista è fondamentale individuare un paradigma di comunicazione
efficiente, in grado cioè di massimizzare lo scambio di informazioni nel modo
più sintetico possibile.\\

In questo contesto, l'approccio genetico puro risulta problematico:
aumentando il numero di Robby presenti sulla mappa, il numero di configurazioni
considerate dall'algoritmo genetico cresce in modo esponenziale, rendendo
computazionalmente dispendiosa la ricerca di corrispondenze efficaci fra
configurazioni in input e possibili azioni. Si è deciso pertanto di tentare
un approccio basato su reti neurali. Tale metodo consente infatti di non
mantenere la corrispondenza diretta tra dati in input e azione da intraprendere,
in quanto tali corrispondenze sono codificate direttamente all'interno della
rete. Fra i vari approcci presenti in letteratura, si è scelto di procedere
mediante una soluzione che evolva la rete in modo puramente genetico.


\section{Strumenti e Tecnologie utilizzate}
\begin{itemize}
 \item Librerie?
 \item Macchine usate
\end{itemize}
\input{core/strumenti.tex}


\section{Progettazione}
\begin{itemize}
 \item Collaborazione(msg)
 \item Reti Neurali
 \item NEAT
 \item Viste globali
\end{itemize}
Al fine di ottenere un'efficiente comunicazione tra i Robby è importante
individuare quali informazioni sia utile scambiare. In quest'ottica si è scelto
di procedere comunicando tra i Robby le rispettive viste locali. A tale
informazione si è scelto inoltre di aggiungere la posizione assoluta del
Robby\footnote{Nel caso in cui la posizione non sia nota, è possibile conoscerla
muovendo il Robby verso un angolo della mappa.}.\\

Si è introdotta in precedenza la volontà di utilizzare una rete neurale per
governare i Robby. A tal fine si è individuato l'approccio \emph{NEAT}
(NeuroEvolution of Augmenting Topologies)\cite{stanley2002evolving}



\section{Implementazione}
\begin{itemize}
 \item Engine
 \item Implementazione del sistema dei messaggi
 \item Viste locali e globali
 \item Rete neurale (neat feed forward)
\end{itemize}
La struttura di base del programma è stata sviluppata in ottica modulare 
utilizzando il linguaggio C++ e la sua libreria standard: a 
run-time è possibile specificare plug-in diversi per gestire la logica dei 
Robby. Tali plug-in devono implementare diverse funzioni: muovere il Robby 
scegliendo la direzione appropriata, generare le strutture dati proprie 
all'algoritmo (es. le reti neurali o i genomi per l'algoritmo genetico puro) e 
liberare la memoria allocata in precedenza. La logica di base del programma 
rimane invece invariata, compresa la funzione di fitness. Tale funzione 
deve tenere conto del numero di lattine raccolte e di eventuali fallimenti 
nelle mosse effettuate (se il Robby si muove nella direzione di un muro, 
oppure 
tenta di raccogliere una lattina su una cella vuota). La fitness è definita 
come segue:
\[\sum\limits_{i=0}^{r} \frac{gc_i}{tc\cdot mn+1}+\frac{sm_i}{t\cdot (tc 
\cdot mn + 1)}\]
Dove:
\begin{itemize}
 \item $r$ è il numero di Robby presenti su ciascuna mappa;
 \item $gc_i$ è il numero di lattine raccolte dall'i-esimo Robby;
 \item $tc$ è il numero totale di lattine per ciascuna mappa;
 \item $mn$ è il numero totale di mappe;
 \item $sm_i$ è il numero di mosse corrette effettuate dall'i-esimo Robby;
 \item $t$ è il numero di turni totali concessi al singolo Robby.
\end{itemize}
Tale formula considera il successo di tutte le mosse possibili come la raccolta 
di una singola lattina ed incrementa di uno il numero totale di lattine. Il 
valore della funzione è compreso fra 0 e 1, dove 0 è il fallimento totale, 
mentre 1 è il successo totale (tutte le lattine raccolte, nessuna mossa 
sbagliata).
\\
Al fine di consentire la comunicazione dei Robby, viene mantenuta una lista 
globale di messaggi. Ciascun messaggio contiene l'identificatore univoco del 
Robby mittente, la propria vista locale, la propria posizione e l'ultima mossa 
da esso effettuata. Tali informazioni vengono poi combinate dal plug-in 
selezionato per scegliere la strategia migliore.
\\
La vista locale ai Robby è una semplice matrice quadrata che viene popolata dal 
core del programma per mostrare ciò che vede il Robby entro una distanza 
specificata parametricamente. Tale vista non è però quadrata: alcune celle che 
sarebbero incluse in una vista quadrata sono più distanti dal Robby rispetto al 
raggio specificato. Per ovviare a tale problema si è scelto di procedere 
utilizzando il seno discreto per distinguere le celle visibili in questo modo:
\[height(n)=\lfloor r \cdot sin(n\cdot\frac{\pi}{2\cdot r})\rfloor \]
Dove:
\begin{itemize}
  \item $r$ è il raggio del cerchio da generare; 
  \item $n$ è il valore intero che rappresenta la distanza tra il punto in 
  esame sull'asse delle ascisse e il centro, ed è quindi compreso fra 1 e r.
\end{itemize}
Si procede iterando n per calcolare quante celle considerare sull'asse delle 
ordinate sopra alla cella individuata da n. Si ottiene così un quarto di 
cerchio, ed è quindi possibile ruotarlo per ottenere l'intero cerchio discreto.
Una volta ottenuto tale cerchio, le celle al suo interno vengono popolate con 
valori interi che ne rappresentano lo stato (vuoto, lattina, Robby). Le celle 
esterne che sono presenti sulla matrice quadrata vengono impostate come 
sconosciute e non sono considerate dall'algoritmo.
\\
Per implementare la vista globale (detta anche known map) prima di ciascun 
turno vengono controllate le viste locali dei Robby e vengono applicate sulla 
mappa globale le informazioni da esse ricavate. Si ottiene così una vista
globale della mappa nella quale le celle sono sconosciute, vuote o contenti
lattine.
\\

Rispetto all'approccio NEAT classico si sono rese necessarie alcune modifiche
per adattare la tecnica al problema in esame. Nell'articolo 
originale\cite{stanley2002evolving} le nuove reti
vengono inizializzate come completamente connesse. Nell'implementazione
presentata si è scelto di procedere con reti inizialmente vuote (prive di
archi) al fine di consentire la ricerca di una rete che abbia una struttura
minimale.\\

Rispetto all'evoluzione topologica, l'algoritmo NEAT genera reti che possono
contenere cicli. Tali reti, che vengono dette ricorrenti (\emph{recurrent neural
network}), offrono vantaggi espressivi soprattutto per problemi nei quali ci sia
la necessità di mantenere una ``memoria" all'interno della rete, che pertanto
non viene azzerata tra due esecuzioni. Tale struttura risulta molto adatta a
gestire un flusso continuo di dati. Tuttavia, rispetto al problema in esame,
tale topologia della rete non offre vantaggi significativi e introduce 
complessità
computazionale nell'attivazione della rete, che deve essere strutturata per
evitare l'occorrenza di cicli infiniti che farebbero altrimenti divergere
l'intero algoritmo. Per ovviare a tale problema si è esteso l'algoritmo NEAT per
impedire la creazione di cicli durante l'evoluzione della rete. A tal fine si è
associata a ciascun nodo una frazione, che indica il livello a cui il nodo
appartiene. Tali valori sono compresi tra 0 e 1 e vengono considerati durante
la \emph{mutazione delle connessioni} e la \emph{mutazione dei nodi}. Durante la
mutazione delle connessioni i livelli dei due nodi adiacenti all'arco da inserire
vengono controllati e, nel caso in cui la connessione sia orientata dal livello
più alto verso il più basso viene invertito il verso dell'arco. Nel caso in cui
i due nodi siano sullo stesso livello l'arco non viene aggiunto. Durante la
mutazione dei nodi il valore del nuovo neurone viene calcolato come la il valore
medio tra i due livelli fra cui verrà inserito (\cref{fig:mutatefraction}).
Tale approccio introduce un ordinamento totale fra livelli, impedendo la
creazione di cicli senza aumentare la complessità computazionale.

\begin{figure}
	\centering
	\begin{tikzpicture}[
		->,
		input node/.style={circle, draw, thick, fill=green!20,minimum size = 7mm},
		hidden node/.style={circle, draw, thick, fill=blue!20,minimum size = 7mm},
		output node/.style={circle, draw, thick, fill=red!20,minimum size = 7mm},
		new node/.style={circle, draw, thick, fill=purple!40,minimum size = 7mm},
		dummy node/.style={circle, thick}
	]
		\node[dummy node] (d1) {};
		\node[dummy node,below of=d1] (d2) {};
		\node[dummy node,below of=d2] (d3) {};
		\node[dummy node,below of=d3] (d4) {};

		\node[input node,right of=d1,xshift=20pt] (i1) {a};
		\node[input node,below of=i1] (i2) {b};
		\node[input node,below of=i2] (i3) {c};
		\node[input node,below of=i3] (i4) {d};

		\node[hidden node,right of=i2,xshift=100pt] (h1) {f};
		\node[hidden node,right of=i3,xshift=100pt] (h2) {g};

		\node[output node,right of=h1,yshift=13pt,xshift=40pt] (o1) {h};
		\node[output node,below of=o1] (o2) {i};
		\node[output node,below of=o2] (o3) {j};

		\node[dummy node,right of=o1, xshift=25pt] (do1) {};
		\node[dummy node,below of=do1] (do2) {};
		\node[dummy node,below of=do2] (do3) {};

		\foreach \i in {1,...,4} {
			\draw[<-] (i\i) -- (d\i) node[above,xshift=0.75cm] {Input \i};
		}

		\draw[->,very thick,red] (i1) -- (h1) {};
		\draw[->] (i2) -- (h1) {};
		\draw[->] (i3) -- (h2) {};
		\draw[->] (i4) -- (h2) {};

		\draw[->] (h1) -- (o3) {};
		\draw[->] (h1) -- (o2) {};
		\draw[->] (h2) -- (o1) {};

		\node[new node,right of=i1,xshift=40pt] (n1) {n};
		\draw[->,very thick,green] (i1) -- (n1) {};
		\draw[->,very thick,green] (n1) -- (h1) {};

		\foreach \o in {1,...,3} {
			\draw[<-] (do\o) -- (o\o) node[above,xshift=1cm] {Output \o};
		}

		\node[dummy node, above of=i1,text width=1cm, align=center]
		(lin) {Level 0};
		\node[dummy node, right of=lin,text width=1cm,align=center,
		xshift=40pt] (lhi) {Level $\frac{1}{4}$};
		\node[dummy node, right of=lin,text width=1cm,align=center,
		xshift=100pt] (lhi) {Level $\frac{1}{2}$};
		\node[dummy node, right of=lhi,text width=1cm,align=center,
		xshift=40pt] (lou) {Level 1};

		
	\end{tikzpicture}
	\caption{Mutazione di un nodo con il meccanismo delle frazioni. Dopo aver
	selezionato l'arco $(a,f)$ viene aggiunto il nodo $n$ e viene posizionato su
	un livello intermedio fra 0 e $\frac{1}{2}$. Il nuovo livello ha quindi
	valore $\frac{1}{4}$. Sono rappresentati in rosso l'arco disabilitato 
	e in verde i nuovi archi.}
	\label{fig:mutatefraction}
\end{figure}

\begin{figure}[H]
\begin{minipage}{.20\textwidth}
\begin{tikzpicture}[
		->,
		input node/.style={circle, draw, thick, fill=green!20,minimum size = 3mm},
		hidden node/.style={circle, draw, thick, fill=blue!20,minimum size = 3mm},
		output node/.style={circle, draw, thick, fill=red!20,minimum size = 3mm},
		dummy node/.style={circle, thick}
]
	\node[dummy node] (pi0) {};
	\node[input node, right of=pi0] (i0) {};
	\draw (pi0) -- (i0);

	\foreach \i in {1,...,5} {
		\auxcount=\i;
		\advance\auxcount by -1;
		\node[dummy node, below of=pi\the\auxcount](pi\i) {};
		\node[input node, below of=i\the\auxcount](i\i) {};

		\draw (pi\i) -- (i\i);
	}

	\node[output node,right of=i1,xshift=1cm] (o0) {};
	\node[dummy node,right of=o0] (po0) {};
	\draw (o0) -- (po0);
	\foreach \o in {1,...,3} {
		\auxcount=\o;
		\advance\auxcount by -1;
		\node[output node, below of=o\the\auxcount](o\o) {};
		\node[dummy node, below of=po\the\auxcount](po\o) {};
		\draw (o\o) -- (po\o);
	}
\end{tikzpicture}
\subcaption{}
\label{fig:mut1}
\end{minipage}
\hspace{.10\textwidth}
\begin{minipage}{.20\textwidth}
\begin{tikzpicture}[
		->,
		input node/.style={circle, draw, thick, fill=green!20,minimum size = 3mm},
		hidden node/.style={circle, draw, thick, fill=blue!20,minimum size = 3mm},
		output node/.style={circle, draw, thick, fill=red!20,minimum size = 3mm},
		dummy node/.style={circle, thick}
]
	\node[dummy node] (pi0) {};
	\node[input node, right of=pi0] (i0) {};
	\draw (pi0) -- (i0);

	\foreach \i in {1,...,5} {
		\auxcount=\i;
		\advance\auxcount by -1;
		\node[dummy node, below of=pi\the\auxcount](pi\i) {};
		\node[input node, below of=i\the\auxcount](i\i) {};

		\draw (pi\i) -- (i\i);
	}

	\node[output node,right of=i1,xshift=1cm] (o0) {};
	\node[dummy node,right of=o0] (po0) {};
	\draw (o0) -- (po0);
	\foreach \o in {1,...,3} {
		\auxcount=\o;
		\advance\auxcount by -1;
		\node[output node, below of=o\the\auxcount](o\o) {};
		\node[dummy node, below of=po\the\auxcount](po\o) {};
		\draw (o\o) -- (po\o);
	}

	\draw (i3) -- (o1);
\end{tikzpicture}
\subcaption{}
\label{fig:mut2}
\end{minipage}
\hspace{.10\textwidth}
\begin{minipage}{.20\textwidth}
\begin{tikzpicture}[
		->,
		input node/.style={circle, draw, thick, fill=green!20,minimum size = 3mm},
		hidden node/.style={circle, draw, thick, fill=blue!20,minimum size = 3mm},
		output node/.style={circle, draw, thick, fill=red!20,minimum size = 3mm},
		dummy node/.style={circle, thick}
]
	\node[dummy node] (pi0) {};
	\node[input node, right of=pi0] (i0) {};
	\draw (pi0) -- (i0);

	\foreach \i in {1,...,5} {
		\auxcount=\i;
		\advance\auxcount by -1;
		\node[dummy node, below of=pi\the\auxcount](pi\i) {};
		\node[input node, below of=i\the\auxcount](i\i) {};

		\draw (pi\i) -- (i\i);
	}

	\node[hidden node,right of=i3,yshift=.5cm] (h0) {};

	\node[output node,right of=i1,xshift=1cm] (o0) {};
	\node[dummy node,right of=o0] (po0) {};
	\draw (o0) -- (po0);
	\foreach \o in {1,...,3} {
		\auxcount=\o;
		\advance\auxcount by -1;
		\node[output node, below of=o\the\auxcount](o\o) {};
		\node[dummy node, below of=po\the\auxcount](po\o) {};
		\draw (o\o) -- (po\o);
	}

	\draw (i3) -- (h0);
	\draw (h0) -- (o1);
\end{tikzpicture}
\subcaption{}
\label{fig:mut3}
\end{minipage}
\caption{Un'evoluzione di esempio di una rete: La rete è inizialmente vuota
(\cref{fig:mut1}),
per poi mutare (tramite una \textbf{mutazione delle connessioni}) un arco tra
due nodi scelti casualmente
(\cref{fig:mut2}),
infine avviene una \textbf{mutazione di nodo} sul
collegamento appena creato
(\cref{fig:mut3})
.}
\centering
\end{figure}



\section{Valutazione e Sviluppi Futuri}
\begin{itemize}
 \item Presentazione dei dati
 \item Migliorare le viste globali
 \item Confronto con algo genetico
\end{itemize}
Per valutare il comportamento della rete si sono utilizzate 100 mappe quadrate 
di lato 10 e di lato 5. Il numero di lattine selezionato è stato scelto in 
maniera proporzionale fra le mappe ed è quindi 50 per le mappe di lato 10 e 13 
per le mappe di lato 5. Gli altri parametri sono stati scelti tramite varie 
esecuzioni di prova per valutarne l'impatto sulla performance. Il numero di 
passi che i Robby possono effettuare è stato selezionato per impedire la 
strategia banale che attraversa l'intera mappa per esplorarla e raccogliere le 
lattine. Da test che non verranno qua analizzati in dettaglio è emerso che con 
un numero sufficiente di round il sistema è in grado di raccogliere fino al 
99\% delle lattine.
\\
Nei test da noi effettuati, contrariamente a quanto atteso, l'utilizzo della 
vista globale non conduce significativi miglioramenti nelle performance del 
sistema. La ragione di tale inefficienza è probabilmente da ricondurre alla 
incapacità della rete di apprendere il significato del valore contenuto 
nelle celle della known map, che assumono valori variabili nel tempo. Le celle 
lontane dal Robby sono infatti sconosciute finché non vengono osservate e tale 
comportamento non permette un apprendimento efficace.
\\
Riportiamo in seguito una visualizzazione grafica del comportamento del 
programma durante 4000 generazioni, su mappe di dimensione 10:
\begin{figure}[H]
\includegraphics[width=.8\textwidth]{img/graph10x10.png}
\caption{}
\end{figure}



\section{Conclusioni}
In conclusione, abbiamo mostrato come sia possibile utilizzare il meccanismo NEAT
per implementare soluzioni al problema presentato. Abbiamo descritto due
approcci diversi per fornire informazioni alla rete neurale. Tramite le reti
neurali è infatti possibile utilizzare informazioni sulla posizione del Robby
da integrare con le viste locali, oppure una vista globale della mappa. Tali
soluzioni risultano facilmente implementabili su reti neurali, ma introducono
problemi di scalabilità non banali per l'algoritmo genetico puro. Rispetto ai
risultati ottenuti, la valutazione non può che essere positiva per quanto
riguarda le viste locali, in quanto i valori di fitness ottenuti risultano
commisurati al numero di turni concessi ai Robby. A differenza della vista
locale la known map ha condotto a risultati deludenti essendo particolarmente
soggetta a massimi locali e \emph{plateau}. Nonostante i buoni risultati vi è
spazio per ulteriori miglioramenti, in particolare per integrare informazioni
globali che migliorino le performance.



\bibliography{core/biblio}
\bibliographystyle{alpha}

\end{document}

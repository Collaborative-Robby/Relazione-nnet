\documentclass[a4paper,10pt,abstracton]{scrartcl}
\usepackage[utf8]{inputenc}
\usepackage[italian]{babel}


\begin{document}

\title{Robby the robot: un approccio collaborativo basato su reti neurali}
\subtitle{Progetto di Fisica dei Sistemi Complessi}

\author{Davide Berardi, Michele Corazza}



\maketitle

\begin{abstract}
parappapa
\end{abstract}

\section{Introduzione}
blablabla

\section{Scenario}
\begin{itemize}
 \item Modello
 \item Robby genetico
\end{itemize}


\section {Obiettivi}
\begin{itemize}
 \item Collaborazione fra robby
 \item Superare l'algoritmo genetico
\end{itemize}


\section{Strumenti e Tecnologie utilizzate}
\begin{itemize}
 \item Librerie?
 \item Macchine usate
\end{itemize}


\section{Progettazione}
\begin{itemize}
 \item Collaborazione(msg)
 \item Reti Neurali
 \item NEAT
 \item Viste globali
\end{itemize}


\section{Implementazione}
\begin{itemize}
 \item Engine
 \item Implementazione del sistema dei messaggi
 \item Viste locali e globali
 \item Rete neurale (neat feed forward)
\end{itemize}


\section{Valutazione e Sviluppi Futuri}
\begin{itemize}
 \item Presentazione dei dati
 \item Migliorare le viste globali
 \item Confronto con algo genetico
\end{itemize}


\section{Conclusioni}

\end{document}

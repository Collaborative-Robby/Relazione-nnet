\documentclass[a4paper,10pt,abstracton]{scrartcl}
\usepackage[utf8]{inputenc}
\usepackage[italian]{babel}


\begin{document}

\title{Robby the robot: un approccio collaborativo basato su reti neurali}
\subtitle{Progetto di Fisica dei Sistemi Complessi}

\author{Davide Berardi, Michele Corazza}



\maketitle

\begin{abstract}
parappapa
\end{abstract}

\section{Introduzione}
blablabla

\section{Scenario}

Il problema affrontato riguarda l'utilizzo di un robot, chiamato Robby, il cui 
scopo è quello di pulire una stanza dalla forma quadrata raccogliendo delle lattine.
La mappa entro la quale il Robby si muove consiste di una griglia 
bidimensionale di celle, che ammette solo posizioni intere.
Le lattine vengono quindi posizionate su celle inizialmente ignote al Robby. Si 
suppone che il Robby sia equipaggiato con sensori in grado di distinguere i muri 
che delimitano l'area esplorabile e di individuare l'eventuale presenza di lattine. La distanza 
massima che il Robby può osservare è un parametro variabile, ma è comunque 
limitata. Altrettanto limitato è il numero di passi discreti che il Robby può 
effettuare; l'azione di raccolta delle lattine viene considerata nel conteggio 
totale dei passi effettuati. La disposizione degli oggetti sulla mappa è 
statica, in altre parole non è possibile che compaiano
ulteriori lattine durante l'esecuzione.
\\
Rispetto ai possibili spostamenti, il Robby può muoversi solamente
in quattro direzioni, non vengono perciò considerate direzioni espresse in 
forma di angolo. Ciascun passo consente perciò al Robby di spostarsi fra due celle 
adiacenti.
\\
Una modalità possibile per affrontare il suddetto problema consiste 
nell'utilizzo di un approccio puramente genetico; si procede creando una 
mappatura fra le possibili configurazioni della vista e le mosse che il Robby 
può effettuare. Tale mappatura viene evoluta tramite crossover e mutazioni 
casuali, al fine di massimizzare l'efficienza del Robby.



\section {Obiettivi}
Un approccio interessante rispetto al suddetto problema riguarda la possibilità
di collaborazione fra più di un Robby sulla stessa mappa. Questa modifica
consente di evidenziare eventuali interazioni complesse fra i Robby. Da questo
punto di vista è fondamentale individuare un paradigma di comunicazione
efficiente, in grado cioè di massimizzare lo scambio di informazioni nel modo
più sintetico possibile.\\

In questo contesto, l'approccio genetico puro risulta problematico:
aumentando il numero di Robby presenti sulla mappa, il numero di configurazioni
considerate dall'algoritmo genetico cresce in modo esponenziale, rendendo
computazionalmente dispendiosa la ricerca di corrispondenze efficaci fra
configurazioni in input e possibili azioni. Si è deciso pertanto di tentare
un approccio basato su reti neurali. Tale metodo consente infatti di non
mantenere la corrispondenza diretta tra dati in input e azione da intraprendere,
in quanto tali corrispondenze sono codificate direttamente all'interno della
rete. Fra i vari approcci presenti in letteratura, si è scelto di procedere
mediante una soluzione che evolva la rete in modo puramente genetico.


\section{Strumenti e Tecnologie utilizzate}
\begin{itemize}
 \item Librerie?
 \item Macchine usate
\end{itemize}


\section{Progettazione}
\begin{itemize}
 \item Collaborazione(msg)
 \item Reti Neurali
 \item NEAT
 \item Viste globali
\end{itemize}


\section{Implementazione}
\begin{itemize}
 \item Engine
 \item Implementazione del sistema dei messaggi
 \item Viste locali e globali
 \item Rete neurale (neat feed forward)
\end{itemize}


\section{Valutazione e Sviluppi Futuri}
\begin{itemize}
 \item Presentazione dei dati
 \item Migliorare le viste globali
 \item Confronto con algo genetico
\end{itemize}


\section{Conclusioni}

\end{document}
